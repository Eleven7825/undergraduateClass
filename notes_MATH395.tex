\documentclass[10pt]{article}

\usepackage{amsmath}
\usepackage{amsfonts}
\usepackage{url}
\usepackage{amsthm}
\usepackage{color}
%\usepackage{stmaryrd}
\usepackage{graphicx}

\newtheorem{mydef}{Definition}
\newtheorem{mylem}{Lemma}
\newtheorem{myass}{Assumption}
\newtheorem{myprop}{Proposition}
\newtheorem{mycor}{Corollary}
\newtheorem{myrem}{Remark}
\newtheorem{mythm}{Theorem}
\newtheorem{myex}{Example}
\newtheorem{myexer}{Exercise}


\newcommand{\bs}[1]{{\boldsymbol #1}}
\newcommand{\hot}[1]{\textcolor{red}{#1}} 
\newcommand{\bsxi}{\bs \xi}

\usepackage[margin=2.5cm]{geometry}
\begin{document}
\title{Mathematical modeling in science, engineering, and economics}
\maketitle

\tableofcontents

\newpage
\section{Network of springs and dashpots}

\subsection{Equations of motion}
We consider a network of springs and dashpots which connect some masses.  Define the following physical parameters:
\begin{align*}
M_k &= \text{mass at node }k \\
S_{jk} &= \text{spring constant between node $j$ and node $k$} \\
D_{jk} &= \text{dashpot constant between node $j$ and node $k$} \\
R^0_{jk} &= \text{resting length of spring between node $j$ and node $k$} 
\end{align*}
For a mass $k$, let $N(k)$ be the index set of masses which are connected to mass $k$.  Let ${\bs X}_k$ and ${\bs U}_k$ be the position and velocity vectors of mass $k$ respectively.  Define a general external force on mass $k$ to be ${\bs F}_k$.  The equations of motion for this network of masses connected by springs and dashpots are
\begin{align}
\label{eq:mot1}
M_k \frac{d {\bs U}_k}{dt} &= \sum_{j \in N(k)} T_{jk} \frac{{\bs X}_j - {\bs X}_k}{\| {\bs X}_j - {\bs X}_k \|} + {\bs F}_k \\
\label{eq:mot2}
\frac{d {\bs X}_k}{dt} &= {\bs U}_k \\
\label{eq:mot3}
T_{jk} &= S_{jk} \left( \|{\bs X}_j - {\bs X}_k\| - R^0_{jk} \right) + D_{jk} \frac{d}{dt} \|{\bs X}_j - {\bs X}_k \|
\end{align}
Equation \eqref{eq:mot1} is a statement of force balance.  Notice the total force on mass $k$ is the sum of forces which each point in the direction from mass $j$ to $k$, given by the unit vector $\frac{{\bs X}_j - {\bs X}_k}{\| {\bs X}_j - {\bs X}_k \|}$, with magnitude $T_{jk}$.  Equation \eqref{eq:mot2} states simply the definition of velocity for mass $k$.  The last equation is the force magnitude, broken into the sum of a spring force magnitude depending on its stretching or compression from its rest length $R_{jk}^0$, and a dashpot force magnitude proportional to the time rate of change of this stretching or compression.

We can simplify the dashpot force magnitude in \eqref{eq:mot3} by considering the time derivative of the square of the displacement and using the chain rule:
\begin{align*}
\frac{d}{dt} \Big(\|{\bs X}_j - {\bs X}_k\|^2\Big) = 2 \|{\bs X}_j - {\bs X}_k\| \, \Big( \frac{d}{dt} \|{\bs X}_j - {\bs X}_k \| \Big).
\end{align*}  
By the product rule, the left hand side becomes:
\begin{align*}
\frac{d}{dt} \Big(\|{\bs X}_j - {\bs X}_k\|^2\Big) = 2({\bs X}_j - {\bs X}_k) \cdot ({\bs U}_j - {\bs U}_k)
\end{align*}
Combining these two equations gives a formula for the time derivative of the displacement:
\begin{align*}
\frac{d}{dt} \|{\bs X}_j - {\bs X}_k \| = \frac{{\bs X}_j - {\bs X}_k}{\|{\bs X}_j - {\bs X}_k\|} \cdot ({\bs U}_j - {\bs U}_k)
\end{align*}
We can then equivalently express the equations of motions as: 
\begin{align}
\label{eq:mot4}
M_k \frac{d {\bs U}_k}{dt} &= \sum_{j \in N(k)} T_{jk} \frac{{\bs X}_j - {\bs X}_k}{\| {\bs X}_j - {\bs X}_k \|} + {\bs F}_k\\
\label{eq:mot5}
\frac{d {\bs X}_k}{dt} &= {\bs U}_k \\
\label{eq:mot6}
T_{jk} &= S_{jk} \left( \|{\bs X}_j - {\bs X}_k\| - R^0_{jk} \right) + D_{jk} \frac{{\bs X}_j - {\bs X}_k}{\|{\bs X}_j - {\bs X}_k\|} \cdot ({\bs U}_j - {\bs U}_k)
\end{align}
Notice that the force magnitude for the dashpot involves the component of the velocity in the direction of spring, which seems to make sense!

The next question we need to answer is how to approximate these equations on a computer.  We are modeling a system which involves space and time, but since the masses are already discrete spatial entities, we just need to worry about approximating the temporal dynamics numerically.  To do this, pick some small, discrete timestep $\Delta t > 0$.  For the {\bf \em forward Euler} method, we use an approximation to the time derivative of the form:
\begin{align*}
\frac{d {\bs U}}{dt}\Big|_{t + \Delta t} \approx \frac{{\bs U}(t+\Delta t ) - {\bs U}(t)}{\Delta t},
\end{align*}
and evaluate the right hand side of the equations of motion at time $t$.  Explicitly, this becomes:
\begin{align*}
M_k \frac{{\bs U}_k(t + \Delta t) - {\bs U}_k(t)}{\Delta t} &= \sum_{j \in N(k)} T_{jk}(t) \frac{{\bs X}_j(t) - {\bs X}_k(t)}{\| {\bs X}_j(t) - {\bs X}_k(t) \|} + {\bs F}_k(t) \\
\frac{{\bs X}_k(t + \Delta t) - {\bs X}_k(t)}{\Delta t} &= {\bs U}_k(t + \Delta t) \\
T_{jk}(t) &= S_{jk} \left( \|{\bs X}_j(t) - {\bs X}_k(t)\| - R^0_{jk} \right) + D_{jk} \frac{{\bs X}_j(t) - {\bs X}_k(t)}{\|{\bs X}_j(t) - {\bs X}_k(t)\|} \cdot ({\bs U}_j(t) - {\bs U}_k(t))
\end{align*}
These equations allow us to first compute ${\bs U}_k(t + \Delta t)$ from the velocities and positions at time $t$, and then use this velocity to compute ${\bs X}_k(t + \Delta t)$. 

We can alternatively evaluate the right hand of the equations of motion at time $t + \Delta t$.  This approach is the {\bf \em backward Euler} method.  To state this method, we just shift the time by $\Delta t$:
\begin{align*}
M_k \frac{{\bs U}_k(t) - {\bs U}_k(t- \Delta t)}{\Delta t} &= \sum_{j \in N(k)} T_{jk}(t) \frac{{\bs X}_j(t) - {\bs X}_k(t)}{\| {\bs X}_j(t) - {\bs X}_k(t) \|} + {\bs F}_k(t)\\
\frac{{\bs X}_k(t) - {\bs X}_k(t - \Delta t)}{\Delta t} &= {\bs U}_k(t) \\
T_{jk}(t) &= S_{jk} \left( \|{\bs X}_j(t) - {\bs X}_k(t)\| - R^0_{jk} \right) + D_{jk} \frac{{\bs X}_j(t) - {\bs X}_k(t)}{\|{\bs X}_j(t) - {\bs X}_k(t)\|} \cdot ({\bs U}_j(t) - {\bs U}_k(t))
\end{align*}
Backward Euler is more complex to implement because the unknowns appear on the right and left hand side of the discretized equations of motion.  This setup results in a nonlinear system to be solved at every timestep, and we can use for example Newton's method to solve this.

\subsection{Modeling the interaction with the ground}

The ground can be generically modeled as a one--dimensional curve in two dimensions, or a two--dimensional surface in three dimensions.  For example, a planar surface for the ground can be specified as:
\begin{align*}
a \, X_1 + b\, X_2 + c \,X_3 = 0
\end{align*}
with normal vector ${\bs n} = [a,b,c]^T$.  A more general equation for the ground takes the form:
\begin{align*}
H(X_1,X_2,X_3) = H({\bs X}) = 0.
\end{align*}
To model some sort of force that the ground exerts on each mass, we need to know when we are above and below it.  For a given mass at location ${\bs X}$ in $\mathbb{R}^2$ or $\mathbb{R}^3$, we say it is {\em above the ground} if $H({\bs X}) > 0$ and {\em below the ground} if $H({\bs X}) < 0$.  The force that the ground exerts on this mass will only be active when the mass is below the ground.

As a modeling choice, we make the magnitude of this force proportional to the distance from the mass to the ground.  Set the mass' location below the ground to be ${\bs X}^0$, and take ${\bs X}$ to be some point on the ground, i.e. $H({\bs X}) = 0$.  Let's take a Taylor expansion:
\begin{align}
\label{eq:te1}
H({\bs X}) - H({\bs X}^0) = \nabla H({\bs X}^0) \cdot ({\bs X} - {\bs X}^0) + O(\|{\bs X} - {\bs X}^0\|^2) 
\end{align}  

A formula to find the distance from a mass at ${\bs X}^0$ to the ground would be the smallest distance $\|{\bs X} - {\bs X}^0\|$ such that $H({\bs X}) = 0$.  This can be stated as an optimization problem:
\begin{align*}
\text{minimize } &\|{\bs X} - {\bs X}^0\| \\
\text{such that } &H({\bs X}) = 0.
\end{align*}
We can equivalently minimize the squared distance $\|{\bs X} - {\bs X}^0\|^2$, which is sometime easier to work with, so let's solve:
\begin{align*}
\text{minimize } &\|{\bs X} - {\bs X}^0\|^2 \\
\text{such that } &H({\bs X}) = 0.
\end{align*}

To solve this optimization problem, it is useful to express the vector ${\bs X} - {\bs X}^0$ in a new basis containing ${\bs V}_1 = \nabla H({\bs X}^0) / \|  \nabla H({\bs X}^0) \|$.  Let's assume we are in $\mathbb{R}^3$ and we pick ${\bs V}_2$ and ${\bs V}_3$ so that $\{ {\bs V}_1,  {\bs V}_2,  {\bs V}_3 \}$ forms an orthonormal basis.  Then we can write:
\begin{align*}
{\bs X} - {\bs X}^0 = \sum_{i=1}^3 \, \Big(({\bs X} - {\bs X}^0) \cdot {\bs V}_i \Big) \, {\bs V}_i, 
\end{align*} 
and
\textbf{\begin{align*}
\|{\bs X} - {\bs X}^0\|^2 = ({\bs X} - {\bs X}^0) \cdot ({\bs X} - {\bs X}^0) = \sum_{i=1}^3 \, |({\bs X} - {\bs X}^0) \cdot {\bs V}_i|^2. 
\end{align*} }
Using ${\bs V}_1 = \nabla H({\bs X}^0) / \|  \nabla H({\bs X}^0) \|$, we can express:
\begin{align}
\label{eq:te2}
\|{\bs X} - {\bs X}^0\|^2 = \left(\frac{({\bs X} - {\bs X}^0) \cdot \nabla H({\bs X}^0)}{\|  \nabla H({\bs X}^0) \|} \right)^2 +  \sum_{i=2}^3 \, |({\bs X} - {\bs X}^0) \cdot {\bs V}_i|^2. 
\end{align} 
Now, use \eqref{eq:te1} to simplify the first term of \eqref{eq:te2}, where we drop the higher order terms in the expansion, and use the constraint that $H({\bs X}) = 0$.  Then the optimization problem can be ``approximated'' by this new one:
\begin{align*}
\text{minimize }&\left(\frac{H({\bs X}^0)}{\|  \nabla H({\bs X}^0) \|}\right)^2 +  \sum_{i=2}^3 \, |({\bs X} - {\bs X}^0) \cdot {\bs V}_i|^2 \\
\text{such that }&H({\bs X}) = 0.
\end{align*} 
Now, try to argue geometrically that we can pick ${\bs X} = {\bs X}^\text{opt}$ so that $H({\bs X}^\text{opt}) = 0$ and $({\bs X}^\text{opt} - {\bs X}^0) \cdot {\bs V}_i = 0$ for $i = 2, 3$.  In fact, we can pick ${\bs X}^\text{opt}$ so that ${\bs X}^\text{opt} - {\bs X}^0$ is parallel to ${\bs V}_1$.  This would be the solution to our new optimization problem, with the minimum distance given by:
\begin{align*}
\| {\bs X}^\text{opt} - {\bs X}^0 \| = \frac{-H({\bs X}^0)}{\|  \nabla H({\bs X}^0) \|}.
\end{align*} 
In summary, we use $-H({\bs X}^0) /\|  \nabla H({\bs X}^0)\|$ to approximate the distance from ${\bs X}^0$ to the ground.  The force from the ground will be proportional to this with some constant $S_\text{ground}$, in the direction of the normal vector to the surface $H({\bs X}) = H({\bs X}^0)$ evaluated at ${\bs X}^0$, which is precisely ${\bs V}_1$.  Putting all this together, the force from the ground is
\begin{align*}
{\bs F}_\text{ground}({\bs X}) = 
\begin{cases}
0 & \text{if }H({\bs X}) \geq 0 \\
-S_\text{ground} \frac{H({\bs X})}{\|\nabla H({\bs X})\|} \frac{\nabla H({\bs X})}{\|\nabla H({\bs X})\|} & \text{if }H({\bs X}) < 0
\end{cases}
\end{align*}

We now wish to describe the effect of friction from the ground.  For this we need the tangential velocity ${\bs U}^{(\text{tan})}$ which can be obtained by projecting the velocity onto the vectors spanning the tangent plane to the ground.  Generically, the projection onto the space of vectors orthogonal to some unit vector ${\bs V}$ is 
\begin{align*}
\mathbb{I} - {\bs V} {\bs V}^T,
\end{align*}
where $\mathbb{I}$ is the identity matrix.  Note that we already have the normal vector at point ${\bs X}$ to the ground, given by  $H({\bs X}) / \|\nabla H({\bs X})\|$.  So, the projection onto the tangent plane to the ground at point ${\bs X}$ is given by the matrix
\begin{align*}
\mathbb{I} -\frac{\nabla H({\bs X})}{\|\nabla H({\bs X})\|}\frac{\nabla H({\bs X})^T}{\|\nabla H({\bs X})\|},
\end{align*}
and then the tangential velocity can be computed by applying this matrix to the velocity vector:
\begin{align*}
{\bs U}^{(\text{tan})} = {\bs U} -\frac{\nabla H({\bs X})}{\|\nabla H({\bs X})\|}\frac{\nabla H({\bs X}) \cdot {\bs U}}{\|\nabla H({\bs X})\|}.
\end{align*}
The frictional force should be proportional to the magnitude of the normal force exerted on the mass, which is precisely $|{\bs F}_\text{ground}({\bs X})|$, with some constant of proportionality $\mu_\text{friction}$.  Its direction should be opposing tangential motion, i.e. in the opposite direction of ${\bs U}^{(\text{tan})}$.  Putting these together, we can explicitly write the frictional force as
\begin{align*}
{\bs F}_\text{friction}({\bs X}) = 
\begin{cases}
0 & \text{if }H({\bs X}) \geq 0 \\
-S_\text{ground}\,\mu_\text{friction} \frac{H({\bs X})}{\|\nabla H({\bs X})\|} \frac{{\bs U}^{(\text{tan})}({\bs X})}{\|{\bs U}^{(\text{tan})}\|} & \text{if }H({\bs X}) < 0
\end{cases}
\end{align*}

\subsection{Rigid body motion}

Going back to the equations of motion for a collections of masses connected by springs, define total mass $M$, position of the center of mass ${\bs X}_\text{cm}$, and velocity of the center of mass ${\bs U}_\text{cm}$ as follows:
\begin{align*}
M = \sum_k M_k, \quad M {\bs X}_\text{cm} = \sum_k M_k {\bs X}_k. \quad M {\bs U}_\text{cm} = \sum_k M_k {\bs U}_k.
\end{align*}
Notice that we have:
\begin{align*}
\frac{d {\bs X}_\text{cm}}{dt} = {\bs U}_\text{cm}.
\end{align*}
To get a differential equation for the velocity of the center of mass, take \eqref{eq:mot4} and sum over $k$.  Notice that the terms:
\begin{align*}
\sum_k \sum_{j \in N(k)} T_{jk} \frac{{\bs X}_j - {\bs X}_k}{\| {\bs X}_j - {\bs X}_k \|} = 0,
\end{align*}
since each term appears twice with opposite sign, and $T_{jk} = T_{kj}$.  So we have derived:
\begin{align*}
M\frac{d {\bs U}_\text{cm}}{dt} = {\bs F} = \sum_k {\bs F}_k.
\end{align*}
To summarize, the equations of motion for the center of mass are:
\begin{align}
\label{eq:rigmot1}
M\frac{d {\bs U}_\text{cm}}{dt} &= {\bs F} = \sum_k {\bs F}_k \\
\label{eq:rigmot2}
\frac{d {\bs X}_\text{cm}}{dt} &= {\bs U}_\text{cm}.
\end{align}
Notice that we haven't made any assumptions about rigidity, and the spring forces are not involved in determining the motion of the center of mass.  We can also derive formula for the angular momentum ${\bs L}$ of the system about its center of mass, who's time derivative is the torque ${\bs \tau}$:
\begin{align*}
\frac{d {\bs L}}{dt} &= {\bs \tau} = \sum_k ({\bs X}_k - {\bs X}_\text{cm}) \times {\bs F}_k \\
{\bs L} &= \sum_k M_k ({\bs X}_k - {\bs X}_\text{cm}) \times ({\bs U}_k - {\bs U}_\text{cm}).
\end{align*}
Now, we make the assumption that our body is rigid.  Then, the motion of each mass in the body at time $t$ is determined by the velocity of the center of mass ${\bs U}_\text{cm}(t)$ and the angular velocity ${\bs \Omega}(t)$ of the whole body about its center of mass.  Explicitly this is given by the following formula:
\begin{align}
\label{eq:rigmot3}
{\bs U}_k(t) = {\bs U}_\text{cm}(t) + {\bs \Omega}(t) \times ({\bs X}_k(t) - {\bs X}_\text{cm}(t)).
\end{align}  
From this formula we can derive a relationship between ${\bs L}$ and ${\bs \Omega}$,
\begin{align}
\label{eq:rigmot4}
{\bs L}(t) = {\bs I}(t) {\bs \Omega}(t),
\end{align}
where ${\bs I}$ is the {\em moment of inertia} tensor defined as follows:
\begin{align*}
{\bs I}(t) &= \sum_k M_k \left( \| \tilde{{\bs X}}_k(t) \|^2 \mathbb{I} - \tilde{{\bs X}}_k(t) \tilde{{\bs X}}_k(t)^T \right) \\
\tilde{{\bs X}}_k(t) &= {\bs X}_k(t) - {\bs X}_\text{cm}(t).
\end{align*}
Just to reiterate, ${\bs I}$ is the moment of inertia tensor and $\mathbb{I}$ is the identity matrix.  It can be thought of as a three-dimensional description of the distribution of mass points in our system, about the center of mass.
\begin{myexer}
Consider four masses in two dimensions, all with same mass, placed on the coordinate axes a distance $r > 0$ from the origin.  Compute the motion of inertia tensor and provide an interpretation.
\end{myexer}
We can argue that the moment of inertia tensor is symmetric positive definite under reasonable assumptions, i.e.
\begin{align*}
{\bs V} \cdot {\bs I}\, {\bs V} > 0 \quad \text{for all }{\bs V} \neq 0,
\end{align*}
which implies it is invertible.  This is a nice fact; given the angular momentum ${\bs L}(t)$ and the moment of inertia ${\bs I}(t)$, at a given time we can compute the angular velocity by solving system \eqref{eq:rigmot4}, and then update the velocities of each mass using \eqref{eq:rigmot3}.  Let ${\bs R}({\bs \Omega}(t),\Delta t)$ denote the orthogonal matrix defining an {\em exact} rotation at angular velocity ${\bs \Omega}(t)$ for a time $\Delta t$.  This matrix is defined later.  Below defines a timestep for a rigid body simulation in detail, from time $t$ to $t + \Delta t$:
\begin{itemize}
\item initial data ${\bs X}_\text{cm}(t)$, ${\bs U}_\text{cm}(t)$, $\ldots, {\bs X}_k(t) \ldots$, ${\bs L}(t)$.
\item given $\tilde{{\bs X}}_k(t)$, compute ${\bs I}(t)$.
\item solve the system ${\bs I}(t){\bs \Omega}(t) = {\bs L}(t)$ for ${\bs \Omega}(t)$.
\item using ${\bs \Omega}(t)$, update $\tilde{{\bs X}}_k(t + \Delta t) = {\bs R}({\bs \Omega}(t), \Delta t)\, \tilde{{\bs X}}_k(t)$.
\item update the center of mass: ${\bs X}_\text{cm}(t + \Delta t) = {\bs X}_\text{cm}(t) + \Delta t \, {\bs U}_\text{cm}(t)$.
\item update the velocity of the center of mass:
\begin{align*}
M \frac{{\bs U}_\text{cm}(t + \Delta t) - {\bs U}_\text{cm}(t)}{\Delta t} = \sum_k {\bs F}_k(t + \Delta t).
\end{align*}
\item update the positions of the individual masses:
\begin{align*}
{\bs X}_k(t+\Delta t) = {\bs X}_\text{cm}(t + \Delta t) + \tilde{{\bs X}}_k(t+\Delta t)
\end{align*}
\item update the angular momentum:
\begin{align*}
\frac{{\bs L}(t + \Delta t) - {\bs L}(t)}{\Delta t} = \sum_k \tilde{{\bs X}}_k(t + \Delta t) \times {\bs F}_k(t + \Delta t).
\end{align*}
\end{itemize}
Now we discuss the construction of the orthogonal matrix ${\bs R}({\bs \Omega}, \Delta t)$, were we have dropped the time $t$ for conciseness.  Given a position vector $\tilde{{\bs X}}$ about the center of mass, this matrix will rotate this vector an amount $\|{\bs \Omega}\|\Delta t$, in radians, in time $\Delta t$.  The part of the vector that will not be affected by the rotation will be its projection onto the unit vector in the direction of ${\bs \Omega}$, namely:
\begin{align*}
\frac{{\bs \Omega}}{\|{\bs \Omega}\|} \frac{{\bs \Omega}^T}{\|{\bs \Omega}\|} \tilde{\bs X} = {\bs P}({\bs \Omega}) \tilde{\bs X}.
\end{align*} 
Draw a picture to see this!!  Expressing
\begin{align*}
\tilde{\bs X} = {\bs P}({\bs \Omega}) \tilde{\bs X} + (\mathbb{I} - {\bs P}({\bs \Omega})) \tilde{\bs X},
\end{align*}
we can see that the part of $\tilde{\bs X}$ we need to rotate is $(\mathbb{I} - {\bs P}({\bs \Omega})) \tilde{\bs X}.$  Using this vector we construct a basis for the plane normal to ${\bs \Omega}$ and through the origin:
\begin{align*}
{\bs V}_1 &=  (\mathbb{I} - {\bs P}({\bs \Omega})) \tilde{\bs X}, \\
{\bs V}_2 &=  \frac{{\bs \Omega}}{\|{\bs \Omega}\|} \times (\mathbb{I} - {\bs P}({\bs \Omega})) \tilde{\bs X} = \frac{{\bs \Omega}}{\|{\bs \Omega}\|} \times \tilde{\bs X}. 
\end{align*}
Again, drawing a picture is a really good way to see this.  Notice that this basis $\{{\bs V}_1, {\bs V}_2\}$ is not necessarily {\em orthonormal}\ldots its important to maintain the correct magnitudes of the vectors here.  Instead $\|{\bs V}_1\| = \|{\bs V}\|_2$, and these norms are not necessarily equal to one.  Now any vector ${\bs V}$ in this plane can be expressed as a linear combination:
\begin{align*}
{\bs V} = \alpha_1 {\bs V}_1 + \alpha_2 {\bs V}_2.
\end{align*}
In particular, the basis vector ${\bs V}_1$, rotated by an amount $\|{\bs \Omega}\|\Delta t$, can be written as 
\begin{align*}
 \cos(\|{\bs \Omega}\|\Delta t){\bs V}_1 + \sin(\|{\bs \Omega}\|\Delta t){\bs V}_2. 
\end{align*}
This rotated vector is precisely the part of $\tilde{\bs X}$ which  rotates about the center of mass.  The orthogonal matrix can be expressed as:
\begin{align*}
{\bs R}({\bs \Omega}, \Delta t) \tilde{\bs X} = {\bs P}({\bs \Omega}) \tilde{\bs X} +  \cos(\|{\bs \Omega}\|\Delta t) (\mathbb{I} - {\bs P}({\bs \Omega})) \tilde{\bs X} + \sin(\|{\bs \Omega}\|\Delta t) \frac{{\bs \Omega}}{\|{\bs \Omega}\|} \times \tilde{\bs X}.
\end{align*}
\begin{myexer}
Show that ${\bs R}({\bs \Omega}, \Delta t)$ is an orthogonal matrix.
\end{myexer}

\newpage
\section{Chemical kinetics}


\end{document}
